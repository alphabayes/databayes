\chapter{Apports du projet et évolutions}

\section{Compétences mobilisées}

En termes de connaissances techniques, mes compétences en programmation objet et en représentation UML m’ont permis de penser efficacement à la structure du projet et à la modélisation de la chaîne de traitement. Par ailleurs, les notions de machine learning vues à l’école m’ont permis de me lancer beaucoup plus rapidement dans la réalisation du code.

Concernant les soft skills, le travail en autonomie a bien sûr été prédominant dans un contexte de télétravail imposé par la crise sanitaire de la COVID-19.

\section{Difficultés rencontrées}

Les difficultées du projet résident surtout dans la compréhension des différentes librairies python, que ce soit pydantic, pandas ou plotly. Par exemple, la structure de l’application requiert qu’on puisse fabriquer des composants graphiques “à la volée”. Or il y avait un problème dans la lecture des callbacks (au moment d’exécuter ces éléments “à la volée” ) pour rendre les visualisations interactives. Ce point n’était pas abordé dans la documentation de dash et il a fallu que je teste plusieurs versions différentes avant de comprendre comment résoudre ce problème.

\section{Compétences acquises}

Les progrès se sont surtout fait sur la partie technique, que ce soit pour l'organisation du code, l'amélioration des \textit{coding skills}, la manipulation des données, la compréhension des modèles de \textit{machine learning}. De plus, le stage a été l'occasion de monter en compétence sur le logiciel de contrôle de version \textbf{git}.

\section{Évolution du projet}

La chaîne peut encore être largement améliorée. Il est tout d’abord possible de compléter les modules déjà existant, en ajoutant de nouvelles mesures de performances, de nouveaux algorithmes de sélection de variables, etc. Ensuite un des enjeux futures est de pouvoir créer un module permettant d’optimiser les paramètres des différents modèles pour obtenir des performances optimales de prédictions. Enfin il reste à styliser et compléter les visualisations déjà existantes.

Une autre évolution du projet évoquée est la possibilité de mettre le projet en open source. Le fort pouvoir collaboratif de la communauté pourrait ainsi permettre au projet de prospérer et d’être utiliser par un grand nombre de personnes.
